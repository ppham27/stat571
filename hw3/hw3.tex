\documentclass[11pt, letterpaper]{article}
\setlength{\parindent}{0in}
\setlength{\textheight}{8.7in}
\setlength{\textwidth}{6.8in}
\setlength{\oddsidemargin}{-0.3in}
\setlength{\evensidemargin}{0.0in}
\addtolength{\topmargin}{-1in}
\setlength{\parskip}{0.1in}

\usepackage{amsmath, amsfonts, color}
\usepackage{bm}
\usepackage{enumerate}
\usepackage{graphicx}
\newcommand*{\justifyheading}{\raggedleft}


\renewcommand{\baselinestretch}{1.0}

\newcommand{\bx}{{\bm x}}
\newcommand{\bX}{{\bm X}}
\newcommand{\by}{{\bm y}}
\newcommand{\bY}{{\bm Y}}
\newcommand{\bW}{{\bm W}}
\newcommand{\bG}{{\bm G}}
\newcommand{\bR}{{\bm R}}
\newcommand{\bZ}{{\bm Z}}
\newcommand{\bV}{{\bm V}}
\newcommand{\bL}{{\bm L}}
\newcommand{\bz}{{\bm z}}
\newcommand{\be}{{\bm e}}
\newcommand{\bgamma}{{\bm \gamma}}
\newcommand{\bbeta}{{\bm \beta}}
\newcommand{\balpha}{{\bm \alpha}}
\newcommand{\bSigma}{{\bm \Sigma}}
\newcommand{\bmu}{{\bm \mu}}
\newcommand{\btheta}{{\bm \theta}}
\newcommand{\bepsilon}{{\bm \epsilon}}
\newcommand{\bone}{{\bm 1}}
\newcommand{\bzero}{{\bm 0}}
\newcommand{\bC}{{\bm C}}
\newcommand{\bI}{{\bm I}}
\newcommand{\bA}{{\bm A}}
\newcommand{\bB}{{\bm B}}
\newcommand{\bQ}{{\bm Q}}
\newcommand{\bS}{{\bm S}}
\newcommand{\bD}{{\bm D}}
\newcommand{\cQ}{\mathcal{Q}}
\newcommand{\cU}{\mathcal{U}}
\newcommand{\cI}{\mathcal{I}}
\newcommand{\cL}{\mathcal{L}}

\newcommand{\beas}{\begin{eqnarray*}}
\newcommand{\eeas}{\end{eqnarray*}}

\newenvironment{equationarrayright}{
                          \begin{eqnarray*}
                          \begin{array}{rcll}
                         }{
                          \end{array}
                          \end{eqnarray*}
                         }
\newcommand{\bear}{\begin{equationarrayright}}
\newcommand{\eear}{\end{equationarrayright}}

\renewcommand\arraystretch{1.3}

\DeclareMathOperator*{\argmin}{arg\,min}

\title{STAT/BIOST 571: Homework 3}
\author{Philip Pham}
\date{\today}

\begin{document}

\maketitle

\section*{Problem 1: Quasilikelihood and semiparametric methods for the general linear model (14 points)}
{\em This question examines the effect of different correlation structures, designs, and sample sizes in fitting a general linear model in a quasi-likelihood and semiparametric framework.  It is also an exercise in writing code systematically; please take care to break the required programming into small tasks, and write individual functions to do each of these tasks.
Please write all code ``by hand'', using matrix algebra and simple moment-based estimators.  
You may find the \texttt{mvtnorm} package helpful.

For the marginal model
\[
E(Y_{ij}|x_{ij}) = \beta_0 + \beta_1 x_{ij},
\]
consider estimation by weighted least squares, where the cluster weights are the inverse of
the estimated cluster covariance matrix.  Calculate quasi-likelihood standard errors as if your assumed form
of the covariance matrix
is known to be correct (even if, in actuality, you have assumed the wrong form of the covariance) and semi-parametric  standard errors using the sandwich estimator that accounts for clustering.
All of the notation follows the lecture notes. 

Throughout, the following are true in the data-generating mechanism
\begin{itemize}
\item $\beta_0=0$, $\beta_1 = 0.5$
\item $\bY_i|\bX_i\sim N(\bX_i \bbeta, \sigma^2 \bR_i)$ with $\sigma^2=1$. 
\end{itemize}
The factors that will vary are
\begin{itemize}
\item The number of clusters is 15, 30, or 60
\item The design: \begin{itemize}
\item Design I has $m_i = 4$, for all clusters. In each cluster, we see $\{x_{i1}, x_{i2},x_{i3},x_{i4}\}=\{7,10,13,16\}$
\item Design II has $m_i=3$ for all clusters.  We see equal numbers of clusters with $\{x_{i1}, x_{i2},x_{i3}\}=\{7,10,13\}, \{7,10,16\}, \{7, 13, 16\}, \textrm{ or }\{10, 13, 16\}$
\end{itemize}
\item The true covariance and the assumed covariance matrices are of the form $\sigma^2 \bR_i$:
\begin{itemize}
\item For the true covariance, consider exchangeable and exponential correlation structures, with $\rho=0.5$ or $\rho=0.9$ (distances between observations in the exponential model based on $x_{ij}$). 
\item For the assumed covariance, consider these and additionally the uncorrelated homoscedastic covariance.  Any covariance parameters should be estimated  using moment-based methods.
\end{itemize}
\end{itemize}


Present results in the table below, and write a paragraph or two summarizing your findings.

\begin{center}
\begin{tabular}{ccr|ccc|ccc|ccc}
&&&\multicolumn{3}{|c}{$\mathbf{SD(\hat\beta_1)}$}&\multicolumn{3}{|c}{$\mathbf{E(\widehat{SE}_{1,QL})}$}&\multicolumn{3}{|c}{$\mathbf{E(\widehat{SE}_{1,sand})}$}\\
&&&\multicolumn{3}{c}{Assumed Corr}&\multicolumn{3}{|c}{Assumed Corr}&\multicolumn{3}{|c}{Assumed Corr}\\
$n$ & Design & True Corr & Uncor & Exch & Expon & Uncor & Exch & Expon & Uncor & Exch & Expon \\ \hline
15 &  I & Exchangeable $\rho=0.5$ & & & & & & & & & \\
15 &  I & Exchangeable $\rho=0.9$ & & & & & & & & & \\
15 &  I &         Exponential $\rho=0.5$ & & & & & & & & & \\
15 &  I &         Exponential $\rho=0.9$ & & & & & & & & & \\ [1ex]
15 & II & Exchangeable $\rho=0.5$ & & & & & & & & & \\
15 & II & Exchangeable $\rho=0.9$ & & & & & & & & & \\
15 & II &         Exponential $\rho=0.5$ & & & & & & & & & \\
15 & II &         Exponential $\rho=0.9$ & & & & & & & & & \\ [2ex]
30 &  I & Exchangeable $\rho=0.5$ & & & & & & & & & \\
30 &  I & Exchangeable $\rho=0.9$ & & & & & & & & & \\
30 &  I &         Exponential $\rho=0.5$ & & & & & & & & & \\
30 &  I &         Exponential $\rho=0.9$ & & & & & & & & & \\ [1ex]
30 & II & Exchangeable $\rho=0.5$ & & & & & & & & & \\
30 & II & Exchangeable $\rho=0.9$ & & & & & & & & & \\
30 & II &         Exponential $\rho=0.5$ & & & & & & & & & \\
30 & II &         Exponential $\rho=0.9$ & & & & & & & & & \\ [2ex]
60 &  I & Exchangeable $\rho=0.5$ & & & & & & & & & \\
60 &  I & Exchangeable $\rho=0.9$ & & & & & & & & & \\
60 &  I &         Exponential $\rho=0.5$ & & & & & & & & & \\
60 &  I &        Exponential $\rho=0.9$ & & & & & & & & & \\ [1ex]
60 & II & Exchangeable $\rho=0.5$ & & & & & & & & & \\
60 & II & Exchangeable $\rho=0.9$ & & & & & & & & & \\
60 & II &         Exponential $\rho=0.5$ & & & & & & & & & \\
60 & II &         Exponential $\rho=0.9$ & & & & & & & & & 
\end{tabular}\\~\\
\textbf{Table of standard deviations of the estimated slopes and 
average of model-based and sandwich-based standard error estimates}
\end{center}
}

\section*{Problem 2: Efficiency of OLS for linear models with correlated data (6 points)}
{\em Review the example on slides 2.34 -- 2.35, which can also be found on pages 60 -- 62 of the Diggle et al. textbook.  We will generalize this example by considering the mean model
\[
E(Y_{ij}) = \beta_0 + \beta_1 x_j
\]
for arbitrary $\bx = (x_1,x_2,\ldots,x_5)$ that is the same for all subjects, but which may or may not
be equal to ${\bm t}=(-2,-1,0,1,2)$ (as is the case in the original version of the example).
Note that the correlation structure is still determined based on $t$, as in the original example, but now the mean model contains $x$ rather than $t$.}
\begin{enumerate}[(a)]
{\em \item Derive a general expression for the relative efficiency of OLS compared to the optimal GLS in estimating $\beta_0$ and $\beta_1$ in this problem.  Your formula should be valid for a homoscedastic exponential covariance matrix with arbitrary
$\rho$ and for arbitrary $\bx$.  That is, derive a general version of the expressions
on the bottom of 2.44 that is valid for any choice of covariates.  Note that it is acceptable for your solution to be written using matrix notation and matrix algebra.}
{\em \item Reproduce the lines in the table on 2.35 that pertain to $\beta_1$ for the following choices of covariate vectors
\beas
\bx&=&(-2,-1,0,1,2)\\
\bx&=&(-1,-2,0,2,1)\\
\bx&=&(0,-1,1,3,2)\\
\bx&=&(0,-1,1,5,2)
\eeas}
{\em \item Explain the key differences between the relative efficiencies you just calculated.  Phrase your answers in a manner that will be understandable by
a quantitavely sophisticated non-statistician (e.g., an epidemiologist collaborator).
}
\end{enumerate} 

\end{document}