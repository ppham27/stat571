\documentclass[11pt, letterpaper]{article}
\setlength{\parindent}{0in}
\setlength{\textheight}{8.7in}
\setlength{\textwidth}{6.8in}
\setlength{\oddsidemargin}{-0.3in}
\setlength{\evensidemargin}{0.0in}
\addtolength{\topmargin}{-1in}
\setlength{\parskip}{0.1in}

\usepackage{amsmath, amsfonts, color}
\usepackage{bm}
\usepackage{enumerate}
\usepackage{graphicx}
\newcommand*{\justifyheading}{\raggedleft}


\renewcommand{\baselinestretch}{1.0}

\newcommand{\bx}{{\bm x}}
\newcommand{\bX}{{\bm X}}
\newcommand{\by}{{\bm y}}
\newcommand{\bY}{{\bm Y}}
\newcommand{\bW}{{\bm W}}
\newcommand{\bG}{{\bm G}}
\newcommand{\bR}{{\bm R}}
\newcommand{\bZ}{{\bm Z}}
\newcommand{\bV}{{\bm V}}
\newcommand{\bL}{{\bm L}}
\newcommand{\bz}{{\bm z}}
\newcommand{\be}{{\bm e}}
\newcommand{\bgamma}{{\bm \gamma}}
\newcommand{\bbeta}{{\bm \beta}}
\newcommand{\balpha}{{\bm \alpha}}
\newcommand{\bSigma}{{\bm \Sigma}}
\newcommand{\bmu}{{\bm \mu}}
\newcommand{\btheta}{{\bm \theta}}
\newcommand{\bepsilon}{{\bm \epsilon}}
\newcommand{\bone}{{\bm 1}}
\newcommand{\bzero}{{\bm 0}}
\newcommand{\bC}{{\bm C}}
\newcommand{\bI}{{\bm I}}
\newcommand{\bA}{{\bm A}}
\newcommand{\bB}{{\bm B}}
\newcommand{\bQ}{{\bm Q}}
\newcommand{\bS}{{\bm S}}
\newcommand{\bD}{{\bm D}}
\newcommand{\cQ}{\mathcal{Q}}
\newcommand{\cU}{\mathcal{U}}
\newcommand{\cI}{\mathcal{I}}
\newcommand{\cL}{\mathcal{L}}

\newcommand{\beas}{\begin{eqnarray*}}
\newcommand{\eeas}{\end{eqnarray*}}

\newenvironment{equationarrayright}{
                          \begin{eqnarray*}
                          \begin{array}{rcll}
                         }{
                          \end{array}
                          \end{eqnarray*}
                         }
\newcommand{\bear}{\begin{equationarrayright}}
\newcommand{\eear}{\end{equationarrayright}}

\renewcommand\arraystretch{1.3}

\DeclareMathOperator*{\argmin}{arg\,min}

\title{STAT/BIOST 571: Homework 4}
\author{Philip Pham}
\date{\today}

\begin{document}

\maketitle

\section*{Problem 1: Iterative methods for solving GEE (10 points)} 

{\em In this problem, you will compare
variants on GEE (or general linear model) estimation procedures when applied to the example dental data set for the model on slide 2.71.
Throughout, use a homoscedastic AR-1 working covariance matrix and estimate $\alpha$ and $\sigma^2$
using moment-based estimators.
For each procedure, compute point estimates and robust sandwich-based standard errors that account for clustering.  Where applicable, report how many interations are required to get convergence of all estimates 
($\bbeta$, $\alpha$, and $\sigma^2$) in their 1st/2nd/3rd significant figures.}
\begin{enumerate}[(a)]
{\em \item Use the non-iterative procedure we have employed previously; that is, first estimate $\bbeta$ by OLS, then estimate the covariance parameters based on residuals from the OLS fit, and then restimate $\bbeta$ 
using the updated working covariance matrix. }
{\em \item Iterate the procedure in part (a), as suggested on slide 2.44.}
\end{enumerate}


\end{document}