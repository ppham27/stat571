\documentclass[11pt, letterpaper]{article}
\setlength{\parindent}{0in}
\setlength{\textheight}{8.7in}
\setlength{\textwidth}{6.8in}
\setlength{\oddsidemargin}{-0.3in}
\setlength{\evensidemargin}{0.0in}
\addtolength{\topmargin}{-1in}
\setlength{\parskip}{0.1in}

\usepackage{amsmath, amsfonts, color}
\usepackage{bm}
\usepackage{enumerate}
\usepackage{graphicx}
\newcommand*{\justifyheading}{\raggedleft}


\renewcommand{\baselinestretch}{1.0}

\newcommand{\bx}{{\bm x}}
\newcommand{\bX}{{\bm X}}
\newcommand{\by}{{\bm y}}
\newcommand{\bY}{{\bm Y}}
\newcommand{\bW}{{\bm W}}
\newcommand{\bG}{{\bm G}}
\newcommand{\bR}{{\bm R}}
\newcommand{\bZ}{{\bm Z}}
\newcommand{\bV}{{\bm V}}
\newcommand{\bL}{{\bm L}}
\newcommand{\bz}{{\bm z}}
\newcommand{\be}{{\bm e}}
\newcommand{\bgamma}{{\bm \gamma}}
\newcommand{\bbeta}{{\bm \beta}}
\newcommand{\balpha}{{\bm \alpha}}
\newcommand{\bSigma}{{\bm \Sigma}}
\newcommand{\bmu}{{\bm \mu}}
\newcommand{\btheta}{{\bm \theta}}
\newcommand{\bepsilon}{{\bm \epsilon}}
\newcommand{\bone}{{\bm 1}}
\newcommand{\bzero}{{\bm 0}}
\newcommand{\bC}{{\bm C}}
\newcommand{\bI}{{\bm I}}
\newcommand{\bA}{{\bm A}}
\newcommand{\bB}{{\bm B}}
\newcommand{\bQ}{{\bm Q}}
\newcommand{\bS}{{\bm S}}
\newcommand{\bD}{{\bm D}}
\newcommand{\cQ}{\mathcal{Q}}
\newcommand{\cU}{\mathcal{U}}
\newcommand{\cI}{\mathcal{I}}
\newcommand{\cL}{\mathcal{L}}

\newcommand{\beas}{\begin{eqnarray*}}
\newcommand{\eeas}{\end{eqnarray*}}

\newenvironment{equationarrayright}{
                          \begin{eqnarray*}
                          \begin{array}{rcll}
                         }{
                          \end{array}
                          \end{eqnarray*}
                         }
\newcommand{\bear}{\begin{equationarrayright}}
\newcommand{\eear}{\end{equationarrayright}}

\renewcommand\arraystretch{1.3}

\DeclareMathOperator*{\argmin}{arg\,min}

\begin{document}


\Large 
\begin{center}
\bf STAT/BIOST 571: Homework 5
\end{center} 
\normalsize 

To be handed in on Friday, February 22. 
Where solutions require use of \texttt{R}, summarize your findings in a written answer. For each question, write up your solution on your own, using \textbf{full sentences}.

\textbf{IMPORTANT NOTE: Please include your annotated code in an appendix, to show what you did.}
\section*{Problem 1: Sandwich and bootstrap standard error estimates (10 points)}
{\em As on slide 2.76, fit the model
\[
EY_{ij}=\beta_0 +\beta_1(Age_{ij}-8)+\beta_2 Gender_i + \beta_3(Age_{ij} -8)\times Gender _i
\]
to the dental data by using REML, but use a homoscedastic covariance models with no correlation.}
\begin{enumerate}[(a)]
{\em \item Calculate sandwich-based standard error estimates for $\hat{\beta_3}$ that account for clustering by subject.  Write your
own code for this, using matrix algebra.}
{\em \item Calculate bootstrap standard error estimates 
for $\hat{\beta}_3$ by resampling clusters.  Describe the results of some basic diagnostics you can do to provide confidence that
bootstrap intervals are valid for this dataset and that you have simulated a sufficient number 
of draws to accurately approximate true bootstrap intervals?}
{ \em \item Calculate bootstrap standard error estimates for $\hat{\beta}_3$ based on resampling observations
without regard to cluster and resampling both clusters and observations within clusters. }
{\em \item Discuss any differences between your sandwich standard error estimates and the three versions of bootstrap standard errors.}
\end{enumerate} 



\end{document}