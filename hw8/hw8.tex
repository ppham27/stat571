\documentclass[11pt, letterpaper]{article}
\setlength{\parindent}{0in}
\setlength{\textheight}{8.7in}
\setlength{\textwidth}{6.8in}
\setlength{\oddsidemargin}{-0.3in}
\setlength{\evensidemargin}{0.0in}
\addtolength{\topmargin}{-1in}
\setlength{\parskip}{0.1in}

\usepackage{amsmath, amsfonts, color}
\usepackage{bm}
\usepackage{booktabs}
\usepackage{enumerate}
\usepackage{graphicx}
\usepackage{multirow}
\usepackage{pdfpages}
\usepackage{placeins}
\newcommand*{\justifyheading}{\raggedleft}


\renewcommand{\baselinestretch}{1.0}

\newcommand{\bx}{{\bm x}}
\newcommand{\bX}{{\bm X}}
\newcommand{\by}{{\bm y}}
\newcommand{\bY}{{\bm Y}}
\newcommand{\bW}{{\bm W}}
\newcommand{\bG}{{\bm G}}
\newcommand{\bR}{{\bm R}}
\newcommand{\bZ}{{\bm Z}}
\newcommand{\bV}{{\bm V}}
\newcommand{\bL}{{\bm L}}
\newcommand{\bz}{{\bm z}}
\newcommand{\be}{{\bm e}}
\newcommand{\bgamma}{{\bm \gamma}}
\newcommand{\bbeta}{{\bm \beta}}
\newcommand{\balpha}{{\bm \alpha}}
\newcommand{\bSigma}{{\bm \Sigma}}
\newcommand{\bmu}{{\bm \mu}}
\newcommand{\btheta}{{\bm \theta}}
\newcommand{\bepsilon}{{\bm \epsilon}}
\newcommand{\bone}{{\bm 1}}
\newcommand{\bzero}{{\bm 0}}
\newcommand{\bC}{{\bm C}}
\newcommand{\bI}{{\bm I}}
\newcommand{\bA}{{\bm A}}
\newcommand{\bB}{{\bm B}}
\newcommand{\bQ}{{\bm Q}}
\newcommand{\bS}{{\bm S}}
\newcommand{\bD}{{\bm D}}
\newcommand{\cQ}{\mathcal{Q}}
\newcommand{\cU}{\mathcal{U}}
\newcommand{\cI}{\mathcal{I}}
\newcommand{\cL}{\mathcal{L}}

\newcommand{\beas}{\begin{eqnarray*}}
\newcommand{\eeas}{\end{eqnarray*}}

\newenvironment{equationarrayright}{
                          \begin{eqnarray*}
                          \begin{array}{rcll}
                         }{
                          \end{array}
                          \end{eqnarray*}
                         }
\newcommand{\bear}{\begin{equationarrayright}}
\newcommand{\eear}{\end{equationarrayright}}


\DeclareMathOperator*{\argmin}{arg\,min}

\title{STAT/BIOST 571: Homework 8}
\author{Philip Pham}
\date{\today}

\begin{document}

\maketitle

\section*{Problem 1: GEE and GLMM; interpretation of marginal parameters in logistic regression models; missing data (20 points)}
 
{\em Download the \texttt{fluoride.csv} dataset from the course website.  This dataset contains 3846 observations of fluoride intake for 1279 children, with follow-ups at ages 1.5, 3, 6, and 9 months, but with some observations missing for individual children.  The variable $id$ indexes unique children, $age$ denotes age in months, $income$ is an indicator for maternal income over 30 thousand dollars per year, $fluoride$ is total 
fluoride intake (mg per kg of body weight), and $fl$ is an indicator for $fluoride > 0.05$.
Our primary interest is the relationship between the binary outcome $fl$ and the child's age, potentially
including effect modification by maternal income.  We will fit logistic regression models for the $fl$ outcome with the 
standard mean variance relationship and either a multiplicative interaction 
\begin{equation}
\label{eq:interact}
\mu=expit(\beta_0 + \beta_1 \times age + \beta_2 \times income + \beta_3 \times age \times income)
\end{equation}
or just an intercept and a main effect 
\begin{equation}
\label{eq:nointeract}
\mu=expit(\beta_0 + \beta_1 \times age). 
\end{equation}
In all analyses, we account for correlation within children and assume the data from different children are independent.}
\begin{enumerate}[(a)]
{\em \item  Fit model~(\ref{eq:interact}) using GEE with independence and exchangeable working
correlation models and using a standard GLMM with random intercepts.  Report point estimates and standard error estimates 
for all four regression coefficients and all three model fits in a single table (use robust standard errors for GEE and model-based versions for GLMM).}

\begin{table}[ht]
  \scriptsize
  \centering
  \begin{tabular}{lrrrrrr}
\toprule
Correlation Structure & \multicolumn{2}{c}{GEE Independent} & \multicolumn{2}{c}{GEE Exchangeable} & \multicolumn{2}{c}{Mixed Model} \\
{} &        Estimate & Standard Error &         Estimate & Standard Error &    Estimate & Standard Error \\
Coefficient            &                 &                &                  &                &             &                \\
\midrule
(Intercept), $\beta_0$ &        0.576457 &       0.117464 &         0.524520 &       0.115018 &    1.171506 &       0.208960 \\
age, $\beta_1$         &       -0.048729 &       0.017763 &        -0.023578 &       0.017634 &   -0.054638 &       0.029439 \\
income, $\beta_2$      &       -0.964447 &       0.148575 &        -0.944916 &       0.145484 &   -2.083890 &       0.269759 \\
age:income, $\beta_3$  &        0.076834 &       0.022462 &         0.061876 &       0.022088 &    0.131501 &       0.036506 \\
\bottomrule
\end{tabular}

  \caption{Model fits of Equation \ref{eq:interact} with different correlation
    structures to the data in \texttt{fluoride.csv}.}
  \label{tab:model_interaction_full_data_fit}
\end{table}

\begin{description}
\item[Solution:] The estimates and standard errors can be found in Table
  \ref{tab:model_interaction_full_data_fit}. Robust sandwich estimates were used
  for standard errors in the GEE model. For the Mixed Model correlation
  structure with random intercepts, model-based standard errors were used.

  Code to fit the models is found in the Appendix.
\end{description}
{\em \item Discuss any differences between the estimated values of $\beta_1$ from your three fitted models.}
\begin{description}
\item[Solution:] In all three models, $\beta_1$ is negative. For those with low
  income, one is more likely to have low fluoride as one ages. See the second
  row of Table \ref{tab:model_interaction_full_data_fit} for the values.

  The effect is most pronounced in the mixed model which lets the intercept term
  vary most freely since each subject has a subject-specific adjustment. The
  exchangeable model encourages within-subject correlation, so it prefers that
  the log odds does not vary much with age.
\end{description}
{\em \item For each of your three fitted models, write a short paragraph summarizing your main findings.  Specifically, give scientifically interpretable statements (including confidence intervals) about the relationships between fluoride intake and age
in children with maternal income greater than 30 thousand dollars per year and
in children with maternal income less than 30 thousand dollars per year.}
\begin{description}
  % latex table generated in R 3.5.2 by xtable 1.8-3 package
% Fri Mar 15 09:35:32 2019
\begin{table}[ht]
\centering
\begingroup\small
\begin{tabular}{rrrr}
  \toprule
 & Point Estimate & 95\% CI lower bound & 95\% CI upper bound \\ 
  \midrule
GEE Independent & -0.04872948 & -0.08354415 & -0.01391481 \\ 
  GEE Exchangeable & -0.02357841 & -0.05814036 & 0.01098355 \\ 
  Mixed Model & -0.05463762 & -0.11233786 & 0.00306262 \\ 
   \bottomrule
\end{tabular}
\endgroup
\caption{Point estimates and confidence intervals for $\beta_1$, which describes how age affects fluoride intake for low-income children.} 
\label{tab:beta_1_intervals}
\end{table}

  % latex table generated in R 3.5.2 by xtable 1.8-3 package
% Fri Mar 15 09:35:32 2019
\begin{table}[ht]
\centering
\begingroup\small
\begin{tabular}{rrrr}
  \toprule
 & Point Estimate & 95\% CI lower bound & 95\% CI upper bound \\ 
  \midrule
GEE Independent & 0.02810417 & 0.00115686 & 0.05505148 \\ 
  GEE Exchangeable & 0.03829750 & 0.01222844 & 0.06436657 \\ 
  Mixed Model & 0.07686298 & 0.03462592 & 0.11910004 \\ 
   \bottomrule
\end{tabular}
\endgroup
\caption{Point estimates and confidence intervals for $\beta_1 + \beta_3$, which describes how age affects fluoride intake for children with maternal income greater than 30 thousand dollars per year.} 
\label{tab:beta_1_3_intervals}
\end{table}

\item[Solution:] $\beta_1$ describes the expected observed change in the log
  odds in children with maternal income less than 30 thousand dollars. Given a
  year of ageing, one would expect to observe a change of $\beta_1$. As
  discussed in the previous section, this quantity is negative. A 95\%
  confidence interval can be found by using the standard error. These results
  can be seen in Table \ref{tab:beta_1_intervals}. For the GEE Exchangeable
  model and Mixed Model, the intervals contain $0$, so only in the GEE
  Indepedent model is the decrease in log odds statistically significant at
  level $\alpha = 0.0t$. Thus, I'd conclude that the effect is quite small if it
  exists at all.

  $\beta_1 + \beta_3$ describes the change in children with maternal income
  greater than 30 thousand dollars per year. This quantity has variance
  \begin{equation}
    \operatorname{var}\left(\beta_1 + \beta_3\right)
    = \operatorname{var}\left(\beta_1\right) + \operatorname{var}\left(\beta_3\right) +
    2\operatorname{cov}\left(\beta_1, \beta_3\right),    
  \end{equation}
  which we can use this to calculate confidence interals. Results are in Table
  \ref{tab:beta_1_3_intervals}. The estimates are all positive, so the
  probability of higher fluoride intake increases with age for children with
  higher maternal incomes. None of the confidence intervals contain $0$, so the
  effect is statistically significant at level $\alpha = 0.05$ in all three
  models. In the models that account for the within-subject correlation (GEE
  Exchangeable and Mixed Model), the estimate is larger. The Mixed Model has
  more parameters and better accounts for the correlation and gives the largest
  estimate.
\end{description}
{\em \item Now repeat part (a), but use model~(\ref{eq:nointeract}) instead of~(\ref{eq:interact}) (there are now only two regression coefficients to report per model).}
\begin{table}[ht]
  \scriptsize
  \centering
  \begin{tabular}{lrrrrrr}
\toprule
Correlation Structure & \multicolumn{2}{c}{GEE Independent} & \multicolumn{2}{c}{GEE Exchangeable} & \multicolumn{2}{c}{Mixed Model} \\
{} &        Estimate & Standard Error &         Estimate & Standard Error &    Estimate & Standard Error \\
Coefficient            &                 &                &                  &                &             &                \\
\midrule
(Intercept), $\beta_0$ &       -0.024537 &       0.070636 &        -0.059908 &       0.068872 &   -0.126181 &       0.114182 \\
age, $\beta_1$         &       -0.002281 &       0.010757 &         0.015402 &       0.010418 &    0.028707 &       0.015460 \\
\bottomrule
\end{tabular}

  \caption{Model fits of Equation \ref{eq:nointeract} with different correlation
    structures to the data in \texttt{fluoride.csv}.}
  \label{tab:model_age_full_data_fit}
\end{table}

\begin{description}
\item[Solution:] The point estimates and standard errors of fitting the model in
  Equation \ref{eq:nointeract} can be found in Table
  \ref{tab:model_age_full_data_fit}. Now $\beta_1$ is close to $0$ in all three
  models. Two of the models are slightly greater than $0$, and one is slightly
  less than $0$..
\end{description}
{\em \item Download the dataset \texttt{fluoride.miss.csv} from the course website and repeat the calculations from part (d).  Note \texttt{fluoride.miss.csv} is a subset of \texttt{fluoride.csv}, with more missing data.}
\begin{table}[ht]
  \scriptsize
  \centering
  \begin{tabular}{lrrrrrr}
\toprule
Correlation Structure & \multicolumn{2}{c}{GEE Independent} & \multicolumn{2}{c}{GEE Exchangeable} & \multicolumn{2}{c}{Mixed Model} \\
{} &        Estimate & Standard Error &         Estimate & Standard Error &    Estimate & Standard Error \\
Coefficient            &                 &                &                  &                &             &                \\
\midrule
(Intercept), $\beta_0$ &       -0.165919 &       0.077793 &        -0.172837 &       0.075095 &   -0.362533 &       0.124891 \\
age, $\beta_1$         &        0.008154 &       0.012001 &         0.021671 &       0.011511 &    0.041221 &       0.017344 \\
\bottomrule
\end{tabular}

  \caption{Model fits of Equation \ref{eq:nointeract} with different correlation
    structures to the data in \texttt{fluoride.miss.csv}.}
  \label{tab:model_age_miss_data_fit}
\end{table}
\begin{description}
\item[Solution:] The results of fitting the model in Equation
  \ref{eq:nointeract} to \texttt{fluoride.miss.csv} can be found in Table
  \ref{tab:model_age_miss_data_fit}. Now all three estimates of $\beta_1$ are
  positive.
\end{description}
{\em \item  Discuss the differences between your results in parts (d) and (e).  Speculate
about the missingness mechanism that gave rise to the \texttt{fluoride.miss.csv} dataset and 
explain how this might account for what you observe.  You might find it
helpful to conduct exploratory analyses of the two datasets and to consider your findings from part (a) of this problem.}

\begin{figure}[ht]
  \centering
  \includegraphics{dataset_comparison.pdf}
  \caption{Number of children by maternal income in \texttt{fluoride.csv} (Full)
    versus \texttt{fluoride.miss.csv} (Partial).}
  \label{fig:dataset_comparison}
\end{figure}

\begin{description}
\item[Solution:] In part (d), the estimates are close to $0$. In part (e), the
  estimates are positive. Only the Mixed Model estimate is statistically
  significant, however.

  Recall that children with higher maternal incomes were more likely to see an
  increase in fluoride intake with age. Indeed, see Figure
  \ref{fig:dataset_comparison}. In \texttt{fluoride.miss.csv}, we are missing
  records of children with low maternal income, so the estimate of $\beta_1$ is
  closer to that of children with higher maternal income, that is, more
  positive.
\end{description}

\end{enumerate}

\section*{Appendix}

Code to fit the models is attached on the following pages.

\FloatBarrier

\includepdf[pages=-]{flouride.pdf}

\end{document}

